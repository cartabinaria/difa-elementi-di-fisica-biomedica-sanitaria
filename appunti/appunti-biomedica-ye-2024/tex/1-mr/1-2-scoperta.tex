\section{Scoperta del fenomeno}
L'esperimento che gettò le basi per lo sviluppo della NMR fu l'esperimento di Stern e Gerlach del 1921.
I due scienziati fecero passare un fascio di atomi di argento attraverso un gradiente di campo magnetico e all'uscita il fascio risultò diviso in due.
Poiché gli atomi di argento hanno un elettrone spaiato, si arrivò alla conclusione che il momento magnetico dell'elettrone spaiato può assumere due diversi valori.
Ulteriori sviluppi portarono alla conclusione che ogni nucleo ha spin intero o semintero.

Un primo vero esperimento sulla risonanza magnetica fu quello di Rabi nel 1939 e illustrato nella \figref{fig:rabi}, in cui un fascio molecolare veniva fatto passare attraverso due magneti (A e B) che generavano due gradienti di campo magnetico opposto.
La deflessione generata sul fascio dai due magneti è uguale e contraria, ma se si inserisce un terzo magnete (C) fra i due magneti originari, il quale genera un campo magnetico uniforme oscillante con la giusta frequenza (la frequenza di risonanza delle molecole del fascio), il fascio defletterà vistosamente.

\begin{figure}[htp]
    \centering
    \includegraphics[scale=0.6]{immagini/rabi.png}
    \caption{\textit{Rappresentazione grafica dell'esperimento di Rabi del 1939}.}
    \label{fig:rabi}
\end{figure}

Attualmente, con il termine NMR ci si riferisce a fenomeni che differiscono da quelli osservati da Rabi per due motivi:
\begin{itemize}[label=$-$]
    \item l'effetto sui momenti magnetici è ottenuto impiegando campi a radiofrequenza;
    \item viene utilizzata materia condensata anziché fasci di atomi.
\end{itemize}
I primi due veri esperimenti di NMR furono quelli di Purcell e Bloch del 1946, condotti indipendentemente e contestualmente, rispettivamente su un solido (la paraffina) e su un liquido (l'acqua).