\section{Risonanza magnetica nei mezzi porosi}
La \textbf{risonanza magnetica per fluidi nei mezzi porosi} (MRPM) è un'applicazione della risonanza magnetica atta a studiare la consistenza di materiali porosi negli ambiti della medicina, dei beni culturali, dell'ambiente e della biologia. L'analisi dei materiali porosi viene condotta studiando il segnale NMR dei protoni, presenti nei fluidi contenuti nei pori, nel dominio del tempo. Le tecniche che si applicano nell'ambito della MRPM sono principalmente rilassometria, diffusometria e NMR \textit{single sided}, ma anche tecniche più recenti come la MR \textit{fingerprint} e il \textit{fast field cycling} (FFC).

\subsection{Ambito medico}
In medicina, la MRPM viene utilizzata per studiare la struttura delle ossa, che è notoriamente porosa. In condizioni normali, nell'adulto il tessuto osseo assume forma lamellare, che gli garantisce una notevole resistenza. Il tessuto osseo lamellare si suddivide in:
\begin{itemize}[label=$-$]
    \item \emph{tessuto compatto}: costituisce la parte esterna dell'osso, fornisce rigidità e durezza ed è attraversato da numerosi canali contenenti vasi sanguigni e dotti linfatici;
    \item \emph{tessuto spugnoso}: forma la parte interna dell'osso, conferisce resistenza e si presenta come un un reticolo tridimensionale di trabecole ossee, che viene a delimitare uno spazio labirintico ripieno di midollo osseo.
\end{itemize}
I tempi di rilassamento dell'acqua, sia trasversale sia longitudinale, subiscono variazioni se l'acqua è contenuta in pori; in particolare, più il volume del poro è piccolo più i tempi di rilassamento si accorciano. Questo elemento può essere utilizzato per studiare la porosità di un osso, mediante la valutazione e il confronto dei segnali NMR provenienti dai pori intertrabecolari e intratrabecolari, questi ultimi più piccoli dei primi. Lo studio della porosità delle ossa è fondamentale nella diagnosi dell'osteoporosi, una malattia sistemica dello scheletro che affligge buona parte della popolazione anziana e che comporta una demineralizzazione dell'osso, con conseguente aumento della porosità. La ricerca in quest'ambito punta a sviluppare tecnologie in grado di rendere possibili campagne di \textit{screening} su larga scala e a basso costo.

\subsection{Ambito culturale}
In ambito di protezione e restauro di beni culturali, un esempio di applicazione della MRPM è la valutazione dell'efficacia di prodotti per il trattamento idrofobico delle rocce. Infatti, l'acqua piovana a lungo andare può danneggiare costruzioni e monumenti, sia corrodendone i materiali, per mezzo delle piogge acide, sia infiltrandosi all'interno dei materiali stessi, a causa dei cambiamenti di fase a cui è soggetta l'acqua, i quali possono minare la struttura interna delle rocce. Studiando il profilo NMR di una roccia è possibile quantificare l'assorbimento di acqua in funzione della profondità, valutando così l'efficacia dei trattamenti idrofobici effettuati sulla roccia.

Un'ulteriore applicazione della risonanza magnetica si ha nella valutazione dell'efficacia dei prodotti di consolidamento delle rocce, valutando la penetrazione dei prodotti stessi mediante l'immersione in acqua e la verifica dell'auspicata riduzione della porosità mediante MRPM.

\subsection{Ambito ambientale}
In campo ambientale, una possibile applicazione della MRPM è lo studio (non invasivo) sui cambiamenti di porosità dello scheletro di alcuni coralli come conseguenza dell'acidificazione degli oceani, un fenomeno direttamente collegato all'inquinamento.
È stato osservato che un aumento del pH ambientale porta a un aumento delle dimensioni dei pori nel corallo: questo perché il corallo ha bisogno di accrescere la sua struttura per raggiungere la maturità sessuale e poter riprodursi, ma in un ambiente troppo acido il processo di calcificazione, che è alla base della costruzione dello scheletro, perde di efficacia; di conseguenza il corallo cresce normalmente ma il suo scheletro non si irrobustisce abbastanza, esponendo il corallo stesso a una maggiore fragilità e mortalità.