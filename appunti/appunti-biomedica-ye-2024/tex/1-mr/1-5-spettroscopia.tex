\section{Spettroscopia di risonanza magnetica}
Se si esclude l'ambito sanitario, la spettroscopia è l'applicazione della risonanza magnetica più utilizzata: è impiegata in biologia per studi anche su cellule vive, in campo veterinario su animali vivi precedentemente anestetizzati, e ovviamente anche in campo chimico.
La spettroscopia di risonanza magnetica (MRS) è l'unica tecnica che permette di eseguire studi di biochimica su cellule e tessuti di organismi vivi.
A differenza dell'MRI, la MRS prevede l'acquisizione del segnale proveniente non dai nuclei di $\mathrm{^1H}$ contenuti nelle molecole d'acqua, ma da quelli contenuti nei metaboliti intracellulari.
Di solito, in ambito clinico, la MRS in vivo viene utilizzata per la rivelazione di nuclei di $\mathrm{^1H}$, abbondanti nell'acqua, di $\mathrm{^{31}P}$, presenti nelle molecole di ATP e impiegati nello studio del metabolismo energetico delle cellule, e di $\mathrm{^{13}C}$, utilizzati nello studio del metabolismo del glucosio.

È opportuno fin da subito delimitare il campo di interesse, a livello di fenomeni fisici soggetti a indagine, della spettroscopia di risonanza magnetica.
Finora, i fenomeni maggiormente trattati sono stati quelli paramagnetici, come l'interazione della materia con un campo magnetico statico e l'effetto Zeeman, che sono anche i fenomeni principali.
La MRS, invece, sfrutta i fenomeni secondari, come l'interazione fra dipoli ma soprattutto il diamagnetismo, che si manifesta nello spostamento chimico e nell'interazione scalare.%
\footnote{Il paramagnetismo è una forma di magnetismo che alcuni materiali mostrano solo in presenza di campi magnetici, e si manifesta con una magnetizzazione avente stessa direzione e stesso verso di quelli associati al campo esterno applicato al materiale paramagnetico stesso.
I materiali paramagnetici sono caratterizzati a livello atomico da dipoli magnetici che si allineano con il campo magnetico applicato, venendone debolmente attratti.

I materiali diamagnetici sono caratterizzati dal fatto che la magnetizzazione ha verso opposto rispetto al campo magnetico, quindi questi materiali ne vengono debolmente respinti.
Si tratta di un effetto molto debole di natura quantistica, che diventa trascurabile se il materiale possiede altre proprietà magnetiche come il ferromagnetismo o il paramagnetismo.}

\subsection{Spostamento chimico}
Lo \textbf{spostamento chimico} è una conseguenza della presenza degli elettroni attorno ai nuclei e il loro effetto è quello di schermare in parte il campo magnetico e modificare la frequenza di Larmor dei nuclei.
Il cambiamento di frequenza dipende dalla specifica nuvola di elettroni che circonda un certo nucleo.
Le nuove espressioni per il campo magnetico e la frequenza di Larmor saranno:
\begin{equation}
    B = B_0(1-\sigma)  \hspace{3 pt} \Rightarrow \hspace{3 pt} \nu=\frac{\gamma B_0(1-\sigma)}{2\pi}\,,
    \label{larmor_schermo}
\end{equation}
dove $\sigma$ è lo schermo elettronico. Nell'immagine di sinistra della \figref{fig:tms} si può notare come, nella molecola dell'acetaldeide, il campo magnetico si comporti diversamente nelle due zone dove risiedono i nuclei di $\mathrm{^1H}$:
\begin{itemize}[label=$-$]
    \item il nucleo di idrogeno all'interno del gruppo $-$COH non è schermato, perché il suo elettrone è fortemente attratto dall'ossigeno, di conseguenza il campo percepito dal nucleo sarà circa uguale a $\Vec{B}_0$;
    \item i nuclei del gruppo $-\mathrm{CH_3}$, invece, sono schermati, perché i loro elettroni non vengono attratti da nessun altro nucleo nelle vicinanze, quindi il campo percepito sarà meno intenso.
\end{itemize}

Per calcolare lo spostamento chimico vero e proprio è necessario fare prima una considerazione.
Se si applica a una molecola un campo magnetico di una certa intensità e si registra il segnale proveniente dai nuclei di idrogeno di quella molecola, per quanto è stato appena detto, risulta che questo venga emesso a frequenze diverse.
Se poi si raddoppia l'intensità del campo, tutte le frequenze di emissione del segnale raddoppieranno a loro volta, in base all'\eqref{larmor_schermo}.
Questo effetto è molto scomodo, perché costringe a fornire il dato sulla frequenza di un segnale accompagnato sempre anche dal dato sull'intensità del campo magnetico applicato: c'è bisogno di una scala assoluta.
Per costruirla, bisogna dotarsi di una frequenza di riferimento sulla quale pesare tutte le frequenze che si vogliono esprimere.
Per tale motivo viene definito lo spostamento chimico:
\begin{equation}
    \boxed{\delta = \dfrac{\nu - \nu_{\mathrm{rif}}}{\nu_{\mathrm{rif}}} \times 10^6}
\end{equation}
misurato in parti per milione (ppm), dove $\nu_{\mathrm{rif}}$ è una frequenza di Larmor di riferimento di una molecola arbitraria.
Di solito, la $\nu_{\mathrm{rif}}$ che viene scelta è quella dei protoni del tetrametilsilano (TMS), raffigurato nell'immagine di destra della \figref{fig:tms}, perché la bassa elettronegatività del silicio e la simmetria della molecola fanno sì che i nuclei di carbonio e idrogeno siano schermati nella stessa misura.

\begin{figure}[htp]
\centering
\includegraphics[scale=0.6]{immagini/acetaldeide.png}\quad\includegraphics[scale=0.687]{immagini/tms.png}
\caption{\label{fig:tms} \textit{A sinistra, azione deformante dell'acetaldeide, e in particolare dell'ossigeno, sul campo magnetico; a destra, struttura del tetrametilsilano}.}
\end{figure}

\subsection{Interazione scalare}
L'\textbf{interazione scalare} è un'interazione indiretta tra gli spin nucleari, mediata dalle nuvole elettroniche degli orbitali molecolari.
In breve, lo spin di un nucleo interagisce con la sua nuvola elettronica, polarizzandola, e quest'ultima trasmetterà la perturbazione ai nuclei appartenenti allo stesso orbitale molecolare del primo nucleo.
È, dunque, un'interazione che coinvolge solo atomi direttamente legati in una molecola.
L'hamiltoniana che descrive l'energia dell'interazione scalare dipende dagli spin dei due nuclei accoppiati e da uno scalare \textit{J}, che per il protone assume un valore compreso fra 0 e 18\,Hz e a sua volta dipende dagli angoli di legame fra il nucleo che ci interessa e quello a cui è direttamente legato.
Di conseguenza, l'interazione scalare non dipenderà dal campo magnetico esterno applicato alla molecola ma solo dalla struttura di quest'ultima; quindi, a differenza dello spostamento chimico, non sarà necessario creare una scala assoluta per poter esprimere coerentemente il dato.
L'equazione della magnetizzazione che tiene conto dell'interazione scalare è:
\begin{equation}
    S(t) = S_0\,\mathrm{e}^{\mathrm{i}(\omega_0t + \varphi)}\mathrm{e}^{\mathrm{i} \pi Jt}\mathrm{e}^{-\frac{t}{\mathrm{T_2}}} \,,
    \label{3.6.3}
\end{equation}
dove $\varphi$ è l'angolo diedro caratteristico del legame fra i nuclei, $\omega_0$ è la pulsazione di Larmor e $\mathrm{T_2}$ è il tempo di rilassamento trasversale.
Concentrandoci sull'andamento temporale del segnale, si deduce che si ha il massimo quando $t = \dfrac{1}{2J}$, perciò bisogna scegliere il tempo di eco $T_\mathrm{E}$ sulla base del valore di \textit{J}.

\begin{wrapfigure}{R}{0.495\textwidth}
    \centering
    \includegraphics[width=0.495\textwidth]{immagini/al.png}
    \caption{\textit{Spettro NMR dell'acido lattico}.}
    \label{fig:al}
\end{wrapfigure}

Nella \figref{fig:al} è raffigurato lo spettro NMR dell'acido lattico, una molecola che viene prodotta da alcuni tipi di cellule tumorali che si sviluppano nel cervello.
Si può notare che, scegliendo tempi di eco sempre maggiori, il segnale cambia d'intensità, passando da positivo a negativo e poi di nuovo positivo, indicando quindi un andamento \virgolette{periodico}, come ci si aspetta dall'\eqref{3.6.3}.
Ciò può essere utile se il segnale non è di per sé molto intenso, per poter distinguere meglio il segnale proveniente dalle singole molecole in uno spettro che ne comprende diverse; fare in modo che alcuni di questi segnali siano ribaltati potrebbe rendere la figura più chiara.
In realtà, l'andamento non è realmente periodico, perché si vede che l'ampiezza del segnale nella terza immagine è minore dell'ampiezza del segnale della prima, a causa del termine dovuto al rilassamento trasversale.
Infine, osserviamo che i picchi sulla destra dello spettro dovrebbero essere un unico picco: anche questo è un effetto dell'interazione scalare.
Il fenomeno della separazione dei picchi è complesso e trova spiegazione nella meccanica quantistica, ma è possibile individuare una semplice regola, illustrata nella \figref{fig:picchi}, per prevedere la distanza che separa i picchi, la loro intensità e il loro numero.
Partendo dalla distanza fra i picchi, questa è costante ed è pari a \textit{J} della molecola, quindi dell'ordine degli hertz.
Il numero di picchi osservati è $n+1$, dove \textit{n} è il numero di nuclei vicini con lo stesso spin presenti nella molecola.
Infine, il rapporto fra le intensità dei picchi è fornito dal triangolo di Tartaglia.

L'importanza dello spostamento chimico e dell'interazione scalare per la MRS sta nel fatto che essi permettono di acquisire informazioni molto dettagliate sulle molecole presenti nei tessuti degli organismi, cosa che non è affatto scontata, vista la complessità delle molecole che si possono incontrare.

\begin{figure}[htp]
\centering
\includegraphics[scale=0.75]{immagini/picchi.png}\quad\includegraphics[scale=0.8]{immagini/tartaglia.png}
\caption{\label{fig:picchi} \textit{A sinistra, la regola \virgolette{$n+1$}; a destra, il triangolo di Tartaglia}.}
\end{figure}

\subsection{Sequenze per MRS}
Le due principali sequenze utilizzate in spettroscopia di risonanza magnetica sono la STEAM e la PRESS.

La \underline{STEAM} (\textit{stimulated echo acquisition mode}) è costituita da tre impulsi di 90°, come mostrato nella \figref{fig:steam}: 
\begin{itemize}[label=$-$]
    \item se la magnetizzazione inizialmente è orientata lungo l'asse \textit{z}, il primo impulso la porterà sul piano \textit{xy}, e per tutto il periodo $\frac{T_\mathrm{E}}{2}$ viene attivato il gradiente di selezione lungo \textit{z};
    \item il secondo impulso orienterà la magnetizzazione lungo il semiasse \textit{-z} e contestualmente viene attivato il gradiente di campo magnetico lungo \textit{x};
    \item il terzo impulso porterà la magnetizzazione di nuovo sul piano \textit{xy} e contestualmente viene attivato il gradiente di campo magnetico lungo \textit{y}.
\end{itemize}
La STEAM ha diversi vantaggi: in primo luogo, il tempo di eco può essere reso molto breve, permettendo la rivelazione dei nuclei caratterizzati da valori di $\mathrm{T_2}$ piccoli.
Inoltre, gli impulsi di 90° consentono di ottenere voxel con bordi molto ben definiti e di limitare la radiazione trasferita alla materia d'indagine.

La \underline{PRESS} (\textit{point-resolved spectroscopy}) è costituita da un impulso di 90° e due impulsi di 180°, come mostrato nella \figref{fig:press2}, e il processo è simile a quanto fatto per la STEAM.
La PRESS è caratterizzata da un buon rapporto segnale/rumore, ma per ottenere voxel con bordi ben definiti è necessario impiegare dei gradienti di campo magnetico efficienti (maggiori di 22\,mT/m).
Il principale svantaggio della PRESS sono i tempi di eco inevitabilmente lunghi, con i quali non si riescono a rivelare i nuclei caratterizzati da valori di $\mathrm{T_2}$ piccoli: ciò la rende inutilizzabile, ad esempio, per la rivelazione del $\mathrm{^{31}P}$.
Inoltre, questo influisce anche sulla radiazione trasmessa al campione indagato, che è maggiore.
Nella \figref{fig:press2} è anche mostrato il procedimento di selezione fino al voxel finale: con il primo impulso si seleziona una sezione del campione, con il secondo si seleziona un parallelepipedo molto stretto e con il terzo il voxel finale.

\begin{figure}[htp]
\centering
\includegraphics[scale=0.55]{immagini/steam.png}
\caption{\label{fig:steam} \textit{Sequenza STEAM}.}
\end{figure}

\begin{figure}[htp]
\centering
\includegraphics[scale=0.6]{immagini/press-2.png}
\caption{\label{fig:press2} \textit{Sequenza PRESS con rappresentazione delle sezioni selezionate passaggio per passaggio}.}
\end{figure}

Bisogna, adesso, porsi il problema della soppressione dei segnali che non ci interessa acquisire e che provengono dal campione che analizziamo.
Un caso importante è la soppressione del segnale proveniente dall'acqua, che in alcune zone del cervello, per fare un esempio, è di diversi ordini di grandezza più abbondante di metaboliti che di solito è interessante rivelare, come la creatina, la colina e l'N-acetilaspartato; di conseguenza, in una MRS di protone il segnale proveniente dagli $\mathrm{^1H}$ coprirebbe di gran lunga tutti gli altri segnali.
Il metodo più usato per la soppressione di segnale coprente indesiderato è il metodo CHESS.

Il \underline{CHESS} (\textit{chemical shift selective pulses}) consiste di due passaggi, illustrati nella \figref{fig:chess}.
Per prima cosa si ha l'applicazione di uno o più impulsi selettivi di tipo $\senc{x}$, con frequenza pari alla frequenza di Larmor dei nuclei della molecola di cui si vuole sopprimere il segnale.
L'impulso selettivo porta la magnetizzazione di quella sola molecola dall'asse \textit{z} al piano \textit{xy}.
Il secondo passaggio consiste, invece, nell'applicazione di un forte gradiente, che provoca lo sfasamento degli spin e la scomparsa della magnetizzazione da sopprimere.
Se le molecole di cui si vuole sopprimere il segnale sono di diverso tipo (ad esempio acqua e grassi), basta applicare in sequenza due impulsi alle opportune frequenze e due gradienti. 

\begin{figure}[htp]
\centering
\includegraphics[scale=0.6]{immagini/chess.png}
\caption{\label{fig:chess} \textit{Metodo di soppressione CHESS}.}
\end{figure}

Dopo aver selezionato il singolo voxel con una sequenza STEAM o PRESS e aver soppresso il segnale che non è utile all'indagine, bisogna occuparsi dei metodi che consentono di mappare il volume del campione: i due metodi principali sono la SV e la sua evoluzione, la CSI.

Il metodo \underline{\textit{single voxel}} (SV-MRS) prevede il campionamento di un piccolo volume nella regione di interesse.
Questa tecnica è caratterizzata da un buon rapporto segnale/rumore, tempi di esecuzione rapidi e una buona risoluzione spaziale, ma permette di studiare solo una piccola regione.

Il metodo \underline{\textit{chemical shift imaging}} (CSI-MRS), invece, prevede il campionamento di più voxel, ossia di una regione più grande. Ciò però la rende una tecnica caratterizzata da un rapporto segnale/rumore più basso e da tempi di esecuzione molto più lunghi rispetto alla SV.
Il funzionamento della CSI è fondamentalmente analogo a quello della SV, ma ripetuto per quanti sono i voxel che si vogliono acquisire.
Se immaginiamo di aver acquisito un voxel all'interno dell'area dello scanner e vogliamo acquisirne un altro adiacente, bisogna semplicemente modificare uno dei gradienti $G_x$ o $G_y$, in base alla direzione lungo la quale ci si vuole spostare rispetto al primo voxel acquisito per selezionare il secondo (si veda la \figref{fig:csi2d}).
Se si vuole condurre un'indagine 2D, basta semplicemente continuare a modificare i due gradienti fino a quando non è stata acquisita tutta la sezione.
Se, invece, si vuole estendere l'indagine a più sezioni, fino anche a tutto il volume del campione posto all'interno dello scanner, bisogna modificare il gradiente di selezione lungo \textit{z}, facendo riferimento alla \figref{fig:csi3d}.
Il tempo di acquisizione totale per una 2D-CSI sarà dato da $N_x \times N_y \times T_\mathrm{R}$, mentre per una 3D-CSI sarà $N_x \times N_y \times N_z \times T_\mathrm{R}$, dove $N_i$ è il numero di voxel lungo la generica direzione \textit{i} e $T_\mathrm{R}$ è il tempo di ripetizione.
I tempi di acquisizione, come si può intuire, possono arrivare a essere parecchio lunghi e questo può portare a una diminuzione eccessiva del rapporto segnale/rumore.
Una \virgolette{soluzione} è quella di aumentare il volume dei voxel, in modo da mappare lo stesso volume in meno tempo, ma ciò influisce negativamente sulla risoluzione spaziale: insomma, nella pratica bisogna sempre trovare un compromesso per ottenere dei risultati adatti alla situazione.

\begin{figure}[htp]
\centering
\includegraphics[scale=0.57]{immagini/csi-2d.png}
\caption{\label{fig:csi2d} \textit{Sequenza PRESS 2D-CSI}.}
\end{figure}

\begin{figure}[htp]
\centering
\includegraphics[scale=0.6]{immagini/csi-3d.png}
\caption{\label{fig:csi3d} \textit{Sequenza PRESS 3D-CSI}.}
\end{figure}

\subsection{Acquisizione MRS}
L'acquisizione per la spettroscopia di risonanza magnetica avviene attraverso delle bobine, che possono sia trasmettere sia ricevere il segnale o solo riceverlo; in alcuni casi vengono usate in maniera combinata o, addirittura, possono essere introdotte all'interno del corpo per una migliore acquisizione.

Dopo che il paziente si è posizionato all'interno dello scanner, si esegue per prima cosa un'acquisizione veloce e poco risolta, chiamata \textit{scout image}, che serve solo a verificare che la posizione del paziente sia corretta; la \figref{fig:scout} è un esempio di \textit{scout image} in cui si vede chiaramente che il paziente deve ricollocarsi meglio all'interno dello scanner.

\begin{figure}[htp]
\centering
\includegraphics[scale=0.7]{immagini/scout.png}
\caption{\label{fig:scout} \textit{Scout image di un cervello}.}
\end{figure}

Il passo successivo è acquisire un'immagine generale e ben definita del volume di interesse, per avere una conoscenza generale della morfologia della regione da indagare; un esempio è fornito nella \figref{fig:gen}.

\begin{figure}[htp]
\centering
\includegraphics[scale=0.75]{immagini/gen.png}
\caption{\label{fig:gen} \textit{3D-MRI di un cervello}.}
\end{figure}

A questo punto, bisogna rendere il campo magnetico $\Vec{B}_0$ il più omogeneo possibile, in modo da limitare gli effetti di spostamento chimico non dovuti agli atomi e alle molecole.
Per verificare l'omogeneità del campo magnetico, si esegue una spettroscopia preliminare per acquisire il picco dei protoni nelle molecole d'acqua: se si osserva che l'ampiezza di tale picco è molto grande, significa che la disomogeneità del campo è grande e va migliorata.
Infine, è necessario eliminare il segnale indesiderato con il metodo CHESS, applicando delle bande di saturazione come mostrato nella \figref{fig:bande}.

\begin{figure}[htp]
\centering
\includegraphics[scale=1.2]{immagini/bande.png}
\caption{\label{fig:bande} \textit{Bande di saturazione per la soppressione del segnale di acqua e grassi}.}
\end{figure}

Per concludere, trattiamo brevemente il rapporto segnale/rumore (SNR).
Più volte è già stato detto che aumentando il numero di acquisizioni di uno stesso segnale è possibile aumentare l'SNR.
Il limite di questo approccio è il fatto che i tempi si allungano a dismisura, e la risonanza magnetica è una tecnica che già di per sé è caratterizzata da tempi lunghi.
L'SNR dipende direttamente dal segnale \textit{S} e inversamente dal rumore \textit{N}.
\textit{S} è proporzionale al numero di medie \textit{n}, poiché il segnale di ogni acquisizione si somma coerentemente con quelli di tutte le altre acquisizioni.
\textit{N}, invece, segue una distribuzione casuale, poiché è dovuto al fatto che si lavora a temperatura ambiente e con componenti elettroniche che si riscaldano quando sono percorse da corrente; di conseguenza, \textit{N} è proporzionale alla radice di \textit{n}.
Abbiamo quindi:
\begin{equation}
    \frac{S}{N} \propto \frac{n}{\sqrt{n}} = \sqrt{n}\,.
\end{equation}
Se ne deduce che, all'aumentare delle acquisizioni \textit{n}, l'SNR aumenta come $\sqrt n$.\\
Volendo fare un riepilogo, il rapporto segnale/rumore dipende dai seguenti fattori:
\begin{itemize}[label=$-$]
    \item nucleo, per via del rapporto giromagnetico $\gamma_\mathrm{n}$ che influisce sulla frequenza di Larmor;
    \item volume di interesse, per via del numero di metaboliti in esso presenti (più è grande, più ce ne saranno);
    \item concentrazione di metaboliti all'interno del volume di interesse, per lo stesso motivo del punto precedente;
    \item campo magnetico $\mathrm{B_0}$, poiché influisce sulla frequenza di Larmor;
    \item bobine impiegate, la cui sensibilità influisce sul campo magnetico $B_1$;
    \item numero di acquisizioni \textit{n};
    \item sequenza d'impulsi utilizzata.
\end{itemize}

\subsection{Post-elaborazione}
La post-elaborazione di un'acquisizione consiste di tutte le tecniche messe in atto ad acquisizione finita per evidenziare uno specifico segnale.
In base alla situazione in cui ci si trova, è conveniente eseguire misurazioni di concentrazioni di metaboliti relative o assolute.

Una \underline{misurazione di concentrazione relativa} si occupa di misurare la concentrazione di un metabolita in una specifica regione in rapporto alla concentrazione di un altro metabolita.
Nelle applicazioni cliniche, di solito, questa misurazione è sufficiente; nel caso del cervello, ad esempio, è usuale studiare la concentrazione relativa dell'N-acetilaspartato o della colina rispetto alla creatina.
Tuttavia, nei casi di patologie in cui il metabolismo delle cellule è ridotto, potrebbe capitare di osservare una diminuzione di tutte e tre le molecole suddette in ugual misura: di conseguenza, una misurazione di concentrazione relativa darebbe gli stessi risultati per un tessuto sano e per uno malato.

Una \underline{misurazione di concentrazione assoluta}, invece, si può eseguire scegliendo un campione di riferimento che non sia uno dei metaboliti che possono subire riduzioni in caso di patologie.
Il campione di riferimento può essere interno o esterno, ma è opportuno che sia interno, ovvero che il suo segnale e la sua concentrazione siano stati misurati nella stessa zona in cui si vuole misurare la concentrazione del metabolita che ci interessa.
Di solito, l'acqua si rivela una buona scelta come campione di riferimento, perché restituisce un buon SNR e possiede un'alta concentrazione nei tessuti.
Inoltre, se si parla del cervello, la concentrazione dell'acqua è piuttosto costante, variando leggermente per la materia bianca e per la materia grigia.
Conoscere la concentrazione del campione di riferimento è fondamentale per poter calcolare la concentrazione del metabolita che ci interessa, in quanto:
\begin{equation}
    C_\mathrm{met} \propto C_\mathrm{acq} \frac{S_\mathrm{met}}{S_\mathrm{acq}}\,,
    \label{esterno}
\end{equation}
dove \textit{C} sono le concentrazioni e \textit{S} i segnali.
Con un po' di attenzione ci si accorge, però, che l'\eqref{esterno} ha bisogno di alcune precisazioni.
Infatti, secondo l'\eqref{segnale}, il segnale dipende dai tempi di rilassamento trasversale e longitudinale.

Per far sì che i segnali dell'acqua e del metabolita siano indipendenti da $\mathrm{T_2}$, è necessario acquisirli a  $T_\mathrm{E}$ molto minore di $\mathrm{T_2}$, idealmente nullo.
Infatti, come mostrato nella \figref{fig:meta}, essi decadono in maniera diversa, a causa del fatto che il segnale dell'acqua non è un'esponenziale ma una somma di segnali esponenziali provenienti sia dall'interno sia dall'esterno delle cellule, quindi ogni altra scelta di $T_\mathrm{E}$ comprometterebbe l'acquisizione.
Tuttavia, è impossibile ridurre il tempo di eco fino a renderlo nullo: l'unica soluzione è acquisire i due segnali per diversi tempi di eco e poi eseguire due fit separati, in modo da ottenere il valore del segnale a $T_\mathrm{E} = 0$ sia per l'acqua sia per il metabolita.
Come spiegato in precedenza, se si acquisissero contemporaneamente i due segnali, quello dell'acqua coprirebbe completamente quello del metabolita, perciò sarà necessario eseguire due acquisizioni separate per le due sostanze; per l'acqua ci sarà bisogno di poche ripetizioni, poiché il segnale che restituisce è tipicamente intenso e di buona qualità, mentre per il metabolita, molto meno spazialmente concentrato dell'acqua, l'acquisizione dovrà essere ripetuta diverse volte.
Rimossa la dipendenza da $\mathrm{T_2}$, rimane da eliminare la dipendenza da $\mathrm{T_1}$: per fare ciò, sempre secondo l'\eqref{segnale}, è sufficiente scegliere il tempo di ripetizione $T_\mathrm{R} \geq 5\mathrm{T_1}$ per entrambi i segnali.
Dopo aver eseguito tutti i passaggi precedenti, il segnale acquisito dovrebbe essere grossomodo indipendente dai tempi di rilassamento, e l'\eqref{esterno} può essere migliorata:
\begin{equation}
    \boxed{C_\mathrm{met} = (0,75 \times 55,5 \times 10^3) \times \frac{S_{0,\mathrm{met}}}{S_{0,\mathrm{acq}}} \times \frac{n_\mathrm{a,acq}}{n_\mathrm{a,met}} \times \frac{n_\mathrm{H,acq}}{n_\mathrm{H,met}}}
\end{equation}
dove 0,75 è la densità dell'acqua, 55,5 è la molarità, $S_0$ sono i segnali indipendenti dai tempi di decadimento, $n_\mathrm{a}$ è il numero di acquisizioni di ciascun segnale e $n_\mathrm{H}$ è il numero di nuclei di idrogeno all'interno delle rispettive molecole.

\begin{figure}[htp]
\centering
\includegraphics[scale=0.5]{immagini/meta.png}
\caption{\label{fig:meta} \textit{Grafico dei segnali di acqua e di un metabolita in funzione del tempo di eco}.}
\end{figure}

Le tecniche di post-elaborazione degli spettri per la quantificazione dell'ampiezza di ciascun segnale sono state principalmente tre.
Inizialmente, fu scelta la tecnica di integrare ogni picco dello spettro, e quindi di calcolare l'area sottostante, ma questa tecnica funziona correttamente solo per spettri con pochi picchi.
Successivamente, si pensò di eseguire sullo spettro un fit multigaussiano, ovvero eseguire un fit gaussiano separatamente per ogni picco dello spettro.
Oggigiorno, invece, si utilizza una tecnica di fit con conoscenza a priori, cioè prima dell'acquisizione si effettua una simulazione delle molecole che ci si aspetta di trovare, o alternativamente si può preparare un campione artificiale con le molecole che ci si aspetta e si sottopone tale campione a spettroscopia.
Poi si sovrappone lo spettro preliminare a quello ottenuto dall'indagine sulla materia da indagare, come mostrato nell'immagine di sinistra della \figref{fig:fit}: se il residuo della sovrapposizione è solo rumore, allora la previsione fatta è corretta e nello spettro si riuscirà a identificare il segnale di ciascun metabolita.

Nell'immagine di destra della \figref{fig:fit} sono riportati gli spettri dei più importanti metaboliti a livello neurologico.
Si può notare come l'unico spettro veramente semplice sia quello della creatina, costituito da soli due picchi, mentre tutti gli altri presentano attività a diverse frequenze e spesso si sovrappongono.
Se non si possiede una conoscenza preliminare dei metaboliti prima di fare l'acquisizione e il fit si rischia di perdere il segnale di una moltitudine di metaboliti.

\begin{figure}[htp]
\centering
\includegraphics[scale=0.7]{immagini/fit.png}\quad\includegraphics[scale=0.7]{immagini/meta-2.png}
\caption{\label{fig:fit} \textit{A sinistra, fit multigaussiano con conoscenza preliminare; a destra, spettri dei principali metaboliti neurologici}.}
\end{figure}

\begin{figure}[htp]
\centering
\includegraphics[scale=0.87]{immagini/gobba.png}
\caption{\label{fig:gobba} \textit{Spettro dovuto a contaminazione da lipidi (in nero) e spettro corretto (in rosso)}.}
\end{figure}

Per concludere, un ulteriore compito della post-elaborazione è quello di accorgersi di errori nell'acquisizione degli spettri.
Nella \figref{fig:gobba} si possono osservare due spettri dello stesso volume di interesse, localizzato nel cervello: quello in nero presenta una gobba anomala dovuta a una non corretta soppressione del segnale proveniente dai lipidi, in questo caso presente nella cute.
Risistemando meglio le bande di saturazione, si riesce a eliminare il segnale lipidico e a ottenere lo spettro corretto in rosso.
Nella \figref{fig:mov}, invece, si osserva uno sdoppiamento di ogni picco nella immagine a), chiaro segno che il paziente si è mosso durante l'acquisizione, di conseguenza le frequenze di emissione che prima erano associate a una certa posizione nello spazio, dopo il movimento saranno associate a una nuova posizione, e la macchina registrerà due frequenze per lo stesso metabolita.

\begin{figure}[htp]
\centering
\includegraphics[scale=1]{immagini/mov.png}
\caption{\label{fig:mov} \textit{Spettri di metaboliti neurologici, di cui a) è stato acquisito in maniera scorretta per un movimento del paziente}.}
\end{figure}

\subsection{Applicazioni}
Nella \figref{fig:spettro} è raffigurato lo spettro di protone dei principali metaboliti neurologici.

L'N-acetilaspartato (NAA) è una molecola presente solo in neuroni e dendriti e rappresenta uno dei marker più importanti per il sistema nervoso.
Il suo picco principale si trova circa a 2\,ppm ed è dovuto ai tre nuclei di idrogeno del gruppo metilico presente nella molecola, mentre il segnale degli altri nuclei di idrogeno cade nelle frequenze dei Glx, tra i 2 e i 3\,ppm.\\
Le molecole che rientrano nel gruppo Glx sono glutammato (Glu), glutammina (Gln) e GABA.
Svolgono un importante ruolo nella neurotrasmissione e le loro molecole sono molto simili, differendo di poco anche dall'NAA, ragion per cui i loro spettri condividono le stesse frequenze.
I Glx non si riescono a distinguere chiaramente con un campo applicato di 1,5\,T, perché l'SNR sarebbe troppo alto.\\
La creatina e la fosfocreatina, indicate con la sigla tCr, sono dei marker per il metabolismo energetico perché agiscono come riserve di gruppi fosfato dai quali viene prodotto ATP.
Il picco principale delle tCr si trova a circa 3 ppm ed è dovuto, come per l'NAA, ai tre nuclei di idrogeno del gruppo metilico, mentre il segnale dei nuclei presenti nel gruppo $\mathrm{-CH_2}$ viene emesso a 3,9\,ppm.\\
La colina, la fosfocolina e la glicerofosforilcolina (tCho) sono associate al metabolismo della membrana cellulare.
Il picco principale delle tCho si trova a 3,2\,ppm ed è dovuto ai nuclei di idrogeno dei tre gruppi metilici legati all'azoto.
Gli spettri delle singole molecole del gruppo sono visibili solo a campi magnetici molto alti.\\
Il mio-inositolo (mI) è un marker dell'accrescimento cellulare e il suo picco, emesso indistintamente dai nuclei di idrogeno non legati all'ossigeno, si trova a 3,6\,ppm.\\
Gli spettri dell'acido lattico (Lac) e dei lipidi (Lip) sono dovuti entrambi ai nuclei di idrogeno presenti nei gruppi metilici e non sono visibili con campi applicati di 1,5\,T, poiché verrebbero coperti dal rumore.
La produzione di acido lattico è associata ad alcuni tipi di tumore, mentre lo spettro dei lipidi può essere dovuto a una non perfetta soppressione del segnale oppure ad alcune patologie.

\begin{figure}[htp]
\centering
\includegraphics[scale=0.63]{immagini/spettro.png}
\caption{\label{fig:spettro} \textit{Spettri di metaboliti neurologici}.}
\end{figure}

\begin{figure}[htp]
\centering
\includegraphics[scale=0.8]{immagini/tum.png}
\caption{\label{fig:tum} \textit{Spettroscopie con diversi tempi di eco, alterate da un tumore cerebrale}.}
\end{figure}

Nella \figref{fig:tum} si possono osservare due spettroscopie eseguite a tempi di eco diversi che possono essere associate a un tumore del tessuto connettivo cerebrale.
Le cose che saltano più all'occhio guardando lo spettro sono la scomparsa quasi totale degli spettri di NAA, Glx e mI, l'ultimo perché caratterizzato da un $\mathrm{T_2}$ piccolo.
Al contrario, si ha un'intensità molto grande per il segnale della colina, proprio perché quel tipo di tumore è soggetto a un accrescimento massiccio, di conseguenza il metabolismo della membrana cellulare è aumentato.
In più, è presente anche acido lattico, visibile soprattutto nell'immagine inferiore in antifase.

Per studiare le patologie con questa metodica è necessario conoscere anche gli spettri dei singoli metaboliti in gruppi di controllo composti da individui sani, e bisogna anche dividere i gruppi di controllo in base a differenze regionali e anagrafiche.
Nella \figref{fig:controllo} è riportato uno studio effettuato su soggetti sani fra i 50 e i 90 anni, per monitorare l'andamento di alcuni metaboliti nella materia bianca del cervello.
Come si può notare, non c'è nessuna correlazione tra l'invecchiamento e la presenza di NAA, Cho e mI, mentre è stato rilevato un aumento significativo della creatina con l'età.

\begin{figure}[htp]
\centering
\includegraphics[scale=0.72]{immagini/controllo.png}
\caption{\label{fig:controllo} \textit{Andamento di Cr, NAA, Cho e mI in gruppi di controllo sani in funzione dell'età}.}
\end{figure}

\begin{figure}[htp]
\centering
\includegraphics[scale=0.55]{immagini/controllo-2.png}
\caption{\label{fig:controllo2} \textit{Spettroscopia della corteccia occipitale in un adulto e in un neonato sani}.}
\end{figure}

Nella \figref{fig:controllo2} si osserva, invece, come la variazione, anche significativa, della concentrazione di metaboliti cerebrali in individui sani non sia esclusivamente correlata a una patologia.
Infatti, nella spettroscopia effettuata sul neonato si vedono dei livelli molto alti di Cho e mI, ma questi possono essere semplicemente associati al veloce accrescimento dei tessuti nei bambini.\\
La concentrazione dei metaboliti può subire variazioni anche considerando zone diverse del cervello.
Ad esempio, tipicamente sia la creatina sia i Glx sono più concentrati nella materia grigia che non nella bianca, mentre l'N-acetilaspartato rimane piuttosto costante.

Nel 2013, sono stati definiti gli ambiti di applicazione in cui la MRS ha più ragione di essere impiegata:

\begin{multicols}{2}
\begin{itemize}[label=$-$]
    \item tumori cerebrali;
    \item lesioni infettive focali;
    \item disordini metabolici;
    \item asfissia perinatale;
    \item sclerosi multipla;
    \item disordini di demielinizzazione;
    \item demenze;
    \item epilessia focale;
    \item sclerosi mesiale temporale;
    \item lesioni ischemiche.
\end{itemize}
\end{multicols}
In particolare nell'ambito dei tumori cerebrali, l'MRS è complementare all'MRI nei casi di:
\begin{itemize}[label=$-$]
    \item diagnosi istologica;
    \item stadiazione tumorale;
    \item guida per biopsia intracraniale;
    \item prognosi;
    \item pianificazione per trattamenti chirurgici e non;
    \item diagnosi differenziale per tumori recidivanti e lesioni post-terapiche.
\end{itemize}

\begin{figure}[htp]
\centering
\includegraphics[scale=0.85]{immagini/glio.png}
\caption{\label{fig:glio} \textit{Imaging e spettroscopia di un cervello affetto da glioblastoma}.}
\end{figure}

\noindent Vediamo alcuni esempi di tumori in cui la spettroscopia e l'\textit{imaging} vengono utilizzati insieme.

Nella \figref{fig:glio} è riportata un'immagine diagnostica in cui si riconosce un glioblastoma, un tumore che colpisce il tessuto connettivo cerebrale, accompagnato da spettroscopia per tre voxel particolari.
Nel voxel A, come ci si potrebbe aspettare, si osserva una diminuzione di NAA e creatina e un incremento di colina, acido lattico e lipidi, il quale, se la soppressione del segnale del tessuto adiposo cutaneo è stata eseguita correttamente, non dovrebbe apparire.
Nei voxel B e C, invece, si osserva una quasi completa scomparsa di tutti i metaboliti, a eccezione dell'acido lattico e dei lipidi, il che indica perdita di funzionalità del tessuto, morte cellulare e accelerato metabolismo delle membrane cellulari dovuto alla crescita del tessuto canceroso.

Nella \figref{fig:astro} si osserva un altro tipo di tumore, l'astrocitoma, e anche qui si osserva una concentrazione ridotta di NAA, un aumento di Cho, ma stavolta non è presente segnale proveniente dai lipidi.

\begin{figure}[htp]
\centering
\includegraphics[scale=0.86]{immagini/astro.png}
\caption{\label{fig:astro} \textit{Immagini pesate in $\mathrm{T_2}$ con metodi diversi e spettroscopia di un cervello affetto da astrocitoma}.}
\end{figure}

\begin{figure}[htp]
\centering
\includegraphics[scale=0.96]{immagini/meni.png}
\caption{\label{fig:meni} \textit{Spettroscopia di un voxel di tessuto affetto da meningioma}.}
\end{figure}

Nella \figref{fig:meni} è raffigurato il risultato della spettroscopia di un tessuto affetto da meningioma.
Si osserva come il segnale della colina sia aumentato mentre quello dell'N-acetilaspartato, che si confonde con i Glx, sia ridotto.
Sebbene possa sembrare diversamente, il picco fra 1,4 e 1,5\,ppm non è quello dell'acido lattico ma di un amminoacido, l'alanina (Ala), che è condizione non necessaria ma sufficiente per poter capire che ciò che si sta osservando è un meningioma.
In generale:
\begin{itemize}[label=$-$]
    \item per tumori di basso grado si osserva un aumento di Cho e mI e una diminuzione di NAA e Cr;
    \item per tumori di alto grado si osserva un aumento di Cho, Lip e Lac e una diminuzione di NAA e Cr;
    \item in caso di metastasi si ha lo stesso spettro dei tumori di alto grado ma con sparizione completa dell'NAA;
    \item in caso di meningioma, si osserva un aumento di Cho e Ala e una diminuzione di NAA e Cr.
\end{itemize}

Non tutte le alterazioni degli spettri sono indice di tumori.
Nella \figref{fig:ks} è mostrato il risultato di una spettroscopia eseguita sul cervello di un paziente affetto da sindrome di Kearns-Sayre, che indica una concentrazione aumentata di colina e acido lattico, sebbene il tessuto non presenti alcuna lesione.
Un esempio \virgolette{opposto} in un certo senso, che comporta un forte aumento di concentrazione di NAA e mI e una lieve diminuzione di Cho, è quello della sindrome di Canavan, una malattia che si manifesta già in età neonatale a causa della mutazione di un gene che codifica l'enzima aspartoacilasi, responsabile dell'eliminazione dell'N-acetilaspartato.
Nella \figref{fig:canavan} sono riportati due spettri, riguardanti un soggetto sano e uno malato.

\begin{figure}[htp]
\centering
\includegraphics[scale=0.57]{immagini/ks.png}
\caption{\label{fig:ks} \textit{Spettroscopia di un voxel di cervello affetto da sindrome di Kearns-Sayre}.}
\end{figure}

\vspace{-10 pt}

\begin{figure}[htp]
\centering
\includegraphics[scale=0.78]{immagini/canavan.png}
\caption{\label{fig:canavan} \textit{Spettroscopie sul cervello di un neonato sano e di uno affetto da sindrome di Canavan}.}
\end{figure}

\noindent Un ultimo esempio è costituito dall'encefalopatia ipossico-ischemica perinatale (EII), che si manifesta sotto forma di lesioni nel cervello del neonato a causa di una carenza di ossigeno durante il parto.
Di solito per la diagnosi è sufficiente un elettroencefalogramma o altri esami clinici; tuttavia, a volte è necessario sottoporre il neonato a MRS, di cui un esempio è riportato nella \figref{fig:asfissia}, dove si possono notare una diminuzione di NAA e un aumento di lipidi e acido lattico; la variazione è tanto maggiore quanto più grave è stata l'ischemia.

\begin{figure}[htp]
\centering
\includegraphics[scale=0.7]{immagini/asfissia.png}
\caption{\label{fig:asfissia} \textit{Spettroscopie sul cervello di un neonato sano e di uno affetto da EII}.}
\end{figure}