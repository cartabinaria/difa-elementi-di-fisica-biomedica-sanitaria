\section{Caratteri generali}
La risonanza magnetica nucleare (NMR) è un fenomeno fisico che coinvolge nuclei atomici con spin diverso da zero ed è caratterizzato da energie molto basse; i nuclei coinvolti nella maggior parte delle applicazioni sono nuclei di $\mathrm{^1H}$ (protoni), soprattutto quelli contenuti nelle molecole d'acqua, e tali nuclei hanno una frequenza di risonanza di 42,6\,MHz, la quale cade nel range delle poco energetiche onde radio.
Per questo motivo, il segnale proveniente dalla risonanza dei nuclei viene rivelato in competizione con il rumore termico e la NMR è considerata, in generale, una tecnica d'indagine poco sensibile e caratterizzata da un basso rapporto segnale/rumore.
La NMR, per questo motivo, viene utilizzata nelle situazioni in cui non è richiesto un gran livello di dettagli dell'immagine; in compenso, è una poco invasiva e abbastanza versatile perché multiparametrica, ossia l'intensità del segnale è pesata su diversi parametri che la rendono utilizzabile in diverse situazioni, sia in ambito diagnostico sia di ricerca scientifica.